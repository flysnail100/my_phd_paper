% !Mode:: "TeX:UTF-8"
%# -*- coding:utf-8 -*-

%% 南京大学学位论文的示例文档
%% 作者:njuhan: https://github.com/njuHan
%% 源模版repo: https://github.com/njuHan/njuthesis-nju-thesis-template

\documentclass[winfonts,phd,twoside]{njuthesis}
%% 20220715
%% 试一下我可不可以修改我的github
%% njuthesis 文档类的可选参数有:
%%   winfonts, linuxfonts, macfonts, adobefonts winfonts 选项使得文档使用Windows 系统提供的字体;linuxfonts 选项使得文档使用Linux 系统提供的字体;macfonts 选项使得文档使用Mac 系统提供的字体;adobefonts 选项使得文档使用Adobe提供的OTF中文字体(需自行下载安转)
%%   phd/master/bachelor 选择博士/硕士/学士论文
%%   twoside 或 oneside 指定排版的文档为双面打印或单面打印格式(twoside会使得chapter 章节从奇数页开始,即纸张的正面开始,因此会出现一些空白的页面)
%%   nobackinfo 取消封二页导师签名信息。注意,按照南大的规定,是需要签名页的。



%%%%%%%%%%%%%%%%%%%%%%%%%%%%%%%%%%%%%%%%%%%%%%%%%%%%%%%%%%%%%%%%%%%%%%%%%%%%%%%
% set up labelformat and labelsep for subfigure 详见: http://www.latexstudio.net/archives/8652.html
\captionsetup[subfigure]{labelformat=simple, labelsep=space}

%%%%%%%%%%%%%%%%%%%%%%%%%%%%%%%%%%%%%%%%%%%%%%%%%%%%%%%%%%%%%%%%%%%%%%%%%%%%%%%
% 设置《国家图书馆封面》的内容,仅博士论文才需要填写

% 设置论文按照《中国图书资料分类法》的分类编号
\classification{0175.2}
% 设置论文按照《国际十进分类法UDC》的分类编号
% 该编号可在下述网址查询:http://www.udcc.org/udcsummary/php/index.php?lang=chi
\udc{004.72}
% 国家图书馆封面上的论文标题第一行,不可换行。此属性可选,默认值为通过\title设置的标题。
\nlctitlea{论文标题第一行}
% 国家图书馆封面上的论文标题第二行,不可换行。此属性可选,默认值为空白。
\nlctitleb{论文标题第二行}
% 国家图书馆封面上的论文标题第三行,不可换行。此属性可选,默认值为空白。
\nlctitlec{}
% 导师的单位名称及地址
\supervisorinfo{南京大学计算机科学与技术系~~南京市汉口路22号~~210093}
% 答辩委员会主席
\chairman{张三丰~~教授}
% 第一位评阅人
\reviewera{阳顶天~~教授}
% 第二位评阅人
\reviewerb{张无忌~~副教授}
% 第三位评阅人
\reviewerc{黄裳~~教授}
% 第四位评阅人
\reviewerd{郭靖~~研究员}


%%%%%%%%%%%%%%%%%%%%%%%%%%%%%%%%%%%%%%%%%%%%%%%%%%%%%%%%%%%%%%%%%%%%%%%%%%%%%%%
% 设置论文的中文封面

% 单行论文标题,不可换行
\title{南京大学毕业论文\LaTeX 模板}

% 如果论文标题过长,可以分两行,第一行用\titlea{}定义,第二行用\titleb{}定义,
% 使用以下3行:
%\title{} %用于覆盖单行标题内容为空
%\titlea{长标题第一行}  %第一行标题写这里
%\titleb{长标题第二行用于长标题换行} %第二行标题写这里
% 注意: \title 不能都注释,它用于控制标题选择双行还是单行。\title{}如果内容为空,则编译\titlea{},titleb{}双行标题,否则编译单行标题


% 论文作者姓名
\author{侯冬旭}
% 论文作者联系电话
\telphone{15950598735}
% 论文作者电子邮件地址
\email{dg1823013@smail.nju.edu.cn}
% 论文作者学生证号
\studentnum{xxxxxxx}
% 论文作者入学年份(年级)
\grade{2012}
% 论文作者毕业年份(届), 出版授权书的学位年度
\graduateyear{20xx}
% 导师姓名职称
\supervisor{赵康僆 \quad 教授}
% 导师的联系电话
\supervisortelphone{123}
% 论文作者的学科与专业方向
\major{计算机科学与技术}
% 论文作者的研究方向
\researchfield{分布式计算}
% 论文作者所在院系的中文名称
\department{电子科学与工程学院}
% 论文作者所在学校或机构的名称。此属性可选,默认值为``南京大学''。
\institute{南京大学}
% 论文的提交日期,需设置年、月、日。
\submitdate{xxxx年 xx 月 xx 日}
% 论文的答辩日期,需设置年、月、日。
\defenddate{xxxx年 xx 月 xx 日}
% 论文的定稿日期,需设置年、月、日。
% 此属性可选,若注释\date{},则默认值为最后一次编译时的日期,精确到日。
%\date{2019年5月20日}

%%%%%%%%%%%%%%%%%%%%%%%%%%%%%%%%%%%%%%%%%%%%%%%%%%%%%%%%%%%%%%%%%%%%%%%%%%%%%%%
% 设置论文的英文封面

% 论文的英文标题,不可换行
\englishtitle{\LaTeX \;  NJU thesis template }
% 论文作者姓名的拼音
\englishauthor{Author}
% 导师姓名职称的英文
\englishsupervisor{ Professor}
% 论文作者学科与专业的英文名
\englishmajor{Computer Science and Technology}
% 论文作者所在院系的英文名称
\englishdepartment{Department of Computer Science and Technology}
% 论文作者所在学校或机构的英文名称。此属性可选,默认值为``Nanjing University''。
\englishinstitute{Nanjing University}
% 论文完成日期的英文形式,它将出现在英文封面下方。需设置年、月、日。日期格式使用美国的日期
% 格式,即``Month day, year'',其中``Month''为月份的英文名全称,首字母大写;``day''为
% 该月中日期的阿拉伯数字表示;``year''为年份的四位阿拉伯数字表示。
% 此属性可选,若注释掉\englishdate{},则默认值为最后一次编译时的日期。
% \englishdate{May 20, 2019}

%%%%%%%%%%%%%%%%%%%%%%%%%%%%%%%%%%%%%%%%%%%%%%%%%%%%%%%%%%%%%%%%%%%%%%%%%%%%%%%
% 设置论文的中文摘要

% 设置中文摘要页面的论文标题及副标题的第一行。
% 此属性可选,其默认值为使用|\title|命令所设置的论文标题
\abstracttitlea{标题第一行}
% 设置中文摘要页面的论文标题及副标题的第二行。
% 此属性可选,其默认值为空白
\abstracttitleb{标题第二行用于长标题换行}

%%%%%%%%%%%%%%%%%%%%%%%%%%%%%%%%%%%%%%%%%%%%%%%%%%%%%%%%%%%%%%%%%%%%%%%%%%%%%%%
% 设置论文的英文摘要

% 设置英文摘要页面的论文标题及副标题的第一行。
% 此属性可选,其默认值为使用|\englishtitle|命令所设置的论文标题
\englishabstracttitlea{englishabstracttitlea}
% 设置英文摘要页面的论文标题及副标题的第二行。
% 此属性可选,其默认值为空白
\englishabstracttitleb{nglishabstracttitleb}

%%%%%%%%%%%%%%%%%%%%%%%%%%%%%%%%%%%%%%%%%%%%%%%%%%%%%%%%%%%%%%%%%%%%%%%%%%%%%%
%% 盲审命令,空白字段设置请看 .cls文件 \newcommand*{\blind}
%% 此外,请按照盲审要求自行去掉个人简历、致谢等页面中的个人信息
%\blind

%%%%%%%%%%%%%%%%%%%%%%%%%%%%%%%%%%%%%%%%%%%%%%%%%%%%%%%%%%%%%%%%%%%%%%%%%%%%%%%
\begin{document}

%%%%%%%%%%%%%%%%%%%%%%%%%%%%%%%%%%%%%%%%%%%%%%%%%%%%%%%%%%%%%%%%%%%%%%%%%%%%%%%

% 制作国家图书馆封面(博士学位论文才需要)
%\makenlctitle
% 制作中文封面
\maketitle
% 制作英文封面
\makeenglishtitle


%%%%%%%%%%%%%%%%%%%%%%%%%%%%%%%%%%%%%%%%%%%%%%%%%%%%%%%%%%%%%%%%%%%%%%%%%%%%%%%
% 开始前言部分
\frontmatter

%%%%%%%%%%%%%%%%%%%%%%%%%%%%%%%%%%%%%%%%%%%%%%%%%%%%%%%%%%%%%%%%%%%%%%%%%%%%%%%
% 论文的中文摘要
\begin{abstract}
\lipsum[1-2]

%通过改变链路中子流的个数,分配不同的数据流量给不同的链路。

% 中文关键词。关键词之间用中文全角分号隔开,末尾无标点符号。
\keywords{关键词1 \quad 关键词2 }
\end{abstract}

%%%%%%%%%%%%%%%%%%%%%%%%%%%%%%%%%%%%%%%%%%%%%%%%%%%%%%%%%%%%%%%%%%%%%%%%%%%%%%%
% 论文的英文摘要
\begin{englishabstract}
\lipsum[2]

%Rate adaptation can be implemented by adjusting the number of subflows on each path.

% 英文关键词。关键词之间用英文半角逗号隔开,末尾无符号。
\englishkeywords{keyword1\quad keyword2}
\end{englishabstract}

%%%%%%%%%%%%%%%%%%%%%%%%%%%%%%%%%%%%%%%%%%%%%%%%%%%%%%%%%%%%%%%%%%%%%%%%%%%%%%%
% 论文的前言,应放在目录之前,中英文摘要之后
%
\begin{preface}
\lipsum[1]
\vspace{1cm}
\begin{flushright}
作者\\
20xx年夏于南京大学
\end{flushright}

\end{preface}

%%%%%%%%%%%%%%%%%%%%%%%%%%%%%%%%%%%%%%%%%%%%%%%%%%%%%%%%%%%%%%%%%%%%%%%%%%%%%%%
% 生成论文目录
\tableofcontents

%%%%%%%%%%%%%%%%%%%%%%%%%%%%%%%%%%%%%%%%%%%%%%%%%%%%%%%%%%%%%%%%%%%%%%%%%%%%%%%
% 生成插图清单。如无需插图清单则可注释掉下述语句。
\listoffigures

%%%%%%%%%%%%%%%%%%%%%%%%%%%%%%%%%%%%%%%%%%%%%%%%%%%%%%%%%%%%%%%%%%%%%%%%%%%%%%%
% 生成附表清单。如无需附表清单则可注释掉下述语句。
\listoftables

%%%%%%%%%%%%%%%%%%%%%%%%%%%%%%%%%%%%%%%%%%%%%%%%%%%%%%%%%%%%%%%%%%%%%%%%%%%%%%%
% 开始正文部分
\mainmatter

%%%%%%%%%%%%%%%%%%%%%%%%%%%%%%%%%%%%%%%%%%%%%%%%%%%%%%%%%%%%%%%%%%%%%%%%%%%%%%%
% 学位论文的正文应以《绪论》作为第一章
\chapter{绪论}\label{chapter_introduction}
%绪论部分应该压缩在20页以内(important)
\section{研究背景}
%使用.bib文件管理参考文献引用,引用示例:\cite{BHR12}.\par
%\lipsum[1]
随着网络、信息技术的快速发展,空间通信已浸润到科技、政治、军事、经济等多个领域,为导航定位、远洋和深空探测、科学实验等应用提供支撑与服务,将人类文化、生产、科学等活动拓展到太空空间,为全球通信欠发达地区提供服务,是全球科学家的研究热点。空间通信是卫星通信系统的进一步发展,其核心是卫星通信。在民用方面,随着用户移动性和带宽多媒体业务的增长,以及基于Internet新应用的涌现,基于卫星的空间通信势必成为未来全球移动通信系统的重要组成部分。并随着航空航天技术、小卫星平台技术和载荷技术的迅猛发展,推动了空间通信的需求和空间任务不断增长,促进了空间通信向网络化、信息一体化的空间信息网络发展。空间通信也将为我国应急支援与搜救、国家安全、远洋探测海事通信、航空通信导航定位、气象遥测遥感及智慧城市等军民应用领域提供必要保障和多样化网络通信服务,其他任何通信方式无法替代,是国家经济与军事发展的两个重要战略制高点,在全球范围内引起广泛关注。

从空间任务角度看,空间通信可分为近地和深空。空间通信具有如下优势:
%(\romannumeral1)
\begin{itemize}
	\item 广域覆盖、机动灵活、实施性好。
	\item 利用互联网技术优势,极大地促进了太空、地面信息资源的流动和共享。
	\item 适应未来发展。随着地球资源的消耗和枯竭,太空将是一个新的利益增长点,需构建统一的空间信息网络,满足未来多样化和复杂的航天应用需求。
	\item 综合性强。融合了多种综合性卫星网络,如卫星通信、导航、测控、遥感等,能够实现不同轨道卫星星座优势互补,是解决深空探测和通信的基础。
\end{itemize}

\section{网络仿真技术}\label{net}

\subsection{网络模拟}

\subsection{真实网络实验床}

\subsection{网络仿真}

\subsection{TCP/IP协议空间应用挑战}
从上个世纪60年代计算机网络发展开始,网络协议技术已历半个多世纪的发展,TCP/IP协议体系已成为地面互联网主要的网络架构。TCP/IP协议体系发源于计算机网络,是一种以主机为中心的网络协议体系,IP地址直接对应到主机,主机与主机之间的数据可靠传输采用“端到端原则”。随着移动通信技术的发展,移动互联网的兴起使得IP地址动态变化问题日益显著,通过移动IP技术可以保证节点漫游过程中的网络连接。从2000年左右,针对以点对点通信为基础的TCP/IP网络体系架构中的关键先天缺陷,主要包括可扩展性问题、动态性问题和安全可控性问题等,试图从根本上解决这些制约网络未来发展的问题。在这些研究工作中,研究人员已经提出了信息中心网络(Information Centric Networking, ICN)等多种区别于传统TCP/IP的新型协议体系。在还没有光纤的年代,最早的跨洋网络线路是通过卫星中继实现。卫星通信网络起源于卫星广播系统,物理层、数据链路层协议多采用DVB系列协议。随着第四代(4G) 、 第五代(5G)移动通信技术的发展,卫星通信也已成为4G、5G标准中的重要组成部分。由于互联网应用以地面为主,作为地面互联网在空间的延伸,卫星通信网络主要采用TCP/IP协议体系。然而,起源于计算机网络的传统TCP协议在面临具有较大带宽、较长时延、较高误码率的卫星信道时,其传输性能及效率大打折扣。为解决这一问题主要采用TCP性能增强代理(PEP) 的方式,将空间段与地面段分割开来,空间段使用优化后的TCP协议,可以大大提高传输性能。然而,由于关口站采用PEP方式打破了端到端传输原则,因此无法应用原有网络安全机制,可能给卫星网络带来潜在的安全风险。

因此,可靠传输的TCP协议应用在空间通信中存在很多问题和挑战。首先TCP对于底层传输协议有许多的假设,如必须存在持续的端到端路径,任意收发节点之间的往返延迟(Round-Trip Time, RTT)较小且相对一致。而且地面通信通常端到端时延的数量级为1~10ms,基于光纤的通信链路通常可达微秒级,但常见卫星通信网络端到端的时延达到500ms,对于深空端到端延迟更长,如地火端到端单向延迟最大可达20.8min。TCP/IP同时也要求通信链路误码率及丢包率较低,应用程序无需考虑通信性能等。在误码率方面,通常地面通信网络的误码率BER小于$10^{-9}$,光纤链路误码率甚至可以达到$10^{-12}$,但空间通信其误码率相对较高,$10^{-6}$是星地射频通信系统常见的误码率,而当通信链路急剧恶化的情况下,尤其对于深空通信甚至会达到$10^{-3}$或$10^{-2}$。并且随着信道误码率的增大,分组丢失增加,而TCP协议对此的处理是将其误归因于网络拥塞所致,因此会减小窗口尺寸降低拥塞门限值,降低了链路的利用率,减小网络吞吐量。可见TCP协议在空间这种间歇性连接、低速率、长延迟的通信环境中其性能将急剧恶化,并导致可靠机制及传统的路由协议如OSPF、BGP等无法正常运行。此外,TCP协议的不同拥塞控制机制均基于有线环境,认为网络拥塞是导致丢包的根本原因。为缓解网络拥塞提高效率,采用单一的降低信源发送速率的方法,虽然在有线网络传输中取得了良好效果,但在复杂的空间信道环境下,如此单一的拥塞控制致使TCP协议数据传输出现了很大困难。

随着地面互联网的快速发展和广泛应用,为了缩减航天成本和易于升级,也产生了在空间通信中直接采用IP技术的想法。2001年美国哥达德航天中心开展了名为OMNI(Operating Mission as Nodes on the Internet)的研究项目,研究采用地面商用IP协议实现空间通信,实现用户到低轨卫星的全IP连接,基本满足地面与近地轨道航天器间的信息传输,但却无法满足空间通信需求。因此,虽然TCP/IP协议技术成熟度高、可移植性好,但需针对空间通信环境进行改善。大量研究也证明受到空间传输条件制约和空间节点组网特殊性的影响,不能将地面互联网技术简单照搬至空间通信网络。

\subsection{空间信息网络仿真平台研究}
近年来,随着空间信息网络相关的研究不断深入开展,对空间信息网络中的通信状态进行有效仿真的需求愈发迫切,因此,各国都针对空间网络仿真平台进行了研究,并产生了相应成果。由于各自的关注点和侧重点有所不同,这些仿真平台分别针对空间信息网络中的不同需求进行了设计与实验。

在空间信息网络研究的不同时期,网络仿真实验的方式也有所不同。在最初网络协议的原理、机制和算法的研究阶段,对仿真平台的真实性要求并不十分苛刻,因此,这一阶段的空间信息网络仿真通常通过网络仿真软件进行实现,如NS-2、OPNET、ONE和OMNeT++等。这些网络仿真软件的实现通常是在单台计算机通过纯软件仿真的方式构造通信场景并进行协议仿真,在仿真的灵活性和操作性上都有着很大的优势,使用者只需要使用一台计算机,就可以灵活的配置整个仿真的场景和链路状态等参数,方便对不同场景下不同网络协议的表现进行比较研究。然而这些网络仿真软件的缺点也十分明显:由于都只是在单机中进行模拟,仿真网络的通信环境与真实环境差距较大,仿真结果的真实性很难得到保证。

为了进一步对空间信息网络中的网络协议及算法性能进行准确评估,拥有真实链路场景和仿真数据的实物及半实物仿真平台成为了研究的关键点。

在\cite{1}中,作者搭建并部署了一个基于实体计算机的空间通信网络测试床(SCNT),测试床由四台实体PC组成,构成了发送-中继-接收的端到端结构,并由一台PC对空间链路环境进行模拟。在此基础上,评估了DTN的LTP协议在不同参数配置和不同网络环境下的性能变化。但这个仿真平台的仿真规模太小且功能固定,不能满足平台灵活配置的要求。

在\cite{2}中,作者搭建了一套基于DTN的BP协议并支持多种DTN协议栈实现的仿真平台,同时该仿真平台包含Portable Satellite Simulator (PSS)与CORTEX CRT等实体卫星通信设备,实现更高的仿真度。在此仿真平台基础上,作者对DTN各种协议栈实现及CFDP协议在空间网络环境中的表现进行了测试。该仿真平台的缺陷在于对卫星间迅速变化的拓扑关系缺少很好的仿真,无法仿真出空间信息网络实际通信中节点间链路存在时间短,链路条件变化快的特点。

在\cite{3}中,作者引入SDN技术和虚拟机技术,搭建了一个先进的仿真测试平台。所有节点接入到SDN交换机,实现节点间拓扑关系的控制;利用虚拟机技术实现较大尺度的网络仿真。但是,该平台在设计中缺少对场景中所有仿真节点的统一控制,因此在仿真过程中缺乏灵活性,无法进行不同仿真场景的灵活切换。

在\cite{4}中,作者搭建了一个较大尺度、分布式的网络仿真平台,通过Openstack和Docker技术,使得该仿真平台具有灵活配置,仿真程度高的特点。各仿真节点与主控制器之间形成星型拓扑结构,能对仿真节点进行有效的配置和任务分配;同时,该仿真平台在链路仿真上也有比较完整和高真实度的设计。该仿真平台仍然存在仿真功能单一,未考虑仿真结果的采集分析,对节点管理配置不够灵活等缺陷,并未完成太过复杂的仿真任务。

\section{主要研究内容与创新点}

\subsection{主要研究内容}

\subsection{论文创新点}

\lipsum[1]
\begin{itemize}
\item 一级item
 \begin{itemize}
 \item 二级item
	\begin{itemize}
	\item 三级item

	\end{itemize}

 \end{itemize}
\item 一级item

\end{itemize}



\chapter{相关工作}

\lipsum[1]

\chapter{chapter}

\lipsum[1]

\section{section}\label{sec:rate}
\lipsum[1]
\subsection{subsection}
\lipsum[1]

\begin{figure}[htbp]
  \centering
  \includegraphics[width=0.6\linewidth]{./figure/github.jpg}
  \caption{单图示例}
  \label{fig:system}
\end{figure}

\chapter{算法}

\begin{algorithm}[htbp]
  \caption{算法名字}
  \label{alg:alg1}
  \begin{algorithmic}[1]
        \REQUIRE 这是输入
        \ENSURE 这是输出
        \WHILE {flag}
		      \STATE 这是语句
        \ENDWHILE
  \end{algorithmic}
\end{algorithm}

\chapter{实验验证}

实验硬件设备如图\ref{img:1}所示。
\begin{figure}[htbp]
\begin{minipage}[t]{0.5\textwidth}
\centering
\includegraphics[width=0.8\textwidth]{./figure/github.jpg}
\caption{实验硬件设备总览}
\label{img:1}
\end{minipage}
\begin{minipage}[t]{0.5\textwidth}
\centering
\includegraphics[width=0.8\textwidth]{./figure/github.jpg}
\caption{实验测量示意图}
\label{img:2}
\end{minipage}
\end{figure}

图\ref{fig:sub}所示子图\ref{subfig:a}和子图\ref{subfig:b}。
\begin{figure}[H]
	\begin{subfigure}{.5\textwidth}
		\centering
		\includegraphics[width=0.8\textwidth]{./figure/github.jpg}
		\caption{子图}
		\label{subfig:a}
	\end{subfigure}
	\begin{subfigure}{.5\textwidth}
		\centering
		\includegraphics[width=0.8\textwidth]{./figure/github.jpg}
		\caption{子图}
		\label{subfig:b}
	\end{subfigure}
\caption{子图样例}
\label{fig:sub}
\end{figure}

\chapter{总结与展望}
\lipsum[1]



%%%%%%%%%%%%%%%%%%%%%%%%%%%%%%%%%%%%%%%%%%%%%%%%%%%%%%%%%%%%%%%%%%%%%%%%%%%%%%%
% 致谢,应放在《结论》之后
\begin{acknowledgement}
%thanks
\lipsum[1]

\end{acknowledgement}

%%%%%%%%%%%%%%%%%%%%%%%%%%%%%%%%%%%%%%%%%%%%%%%%%%%%%%%%%%%%%%%%%%%%%%%%%%%%%%%




% 参考文献。应放在\backmatter之前。
% 推荐使用BibTeX,若不使用BibTeX时注释掉下面一句。
%\nocite{*}
%博士论文的参考文献应保证在150篇的样子
\bibliography{sample}


% 附录,必须放在参考文献后,backmatter前
\appendix
\chapter{附录代码}\label{app:1}
\section{main函数}
\begin{lstlisting}[language=C]
int main()
{
	return 0;
}
\end{lstlisting}

%%%%%%%%%%%%%%%%%%%%%%%%%%%%%%%%%%%%%%%%%%%%%%%%%%%%%%%%%%%%%%%%%%%%%%%%%%%%%%%
% 书籍附件
\backmatter
%%%%%%%%%%%%%%%%%%%%%%%%%%%%%%%%%%%%%%%%%%%%%%%%%%%%%%%%%%%%%%%%%%%%%%%%%%%%%%%
% 作者简历与科研成果页,应放在backmatter之后
\begin{resume}
% 论文作者身份简介,一句话即可。
\begin{authorinfo}
\noindent 韦小宝,男,汉族,1985年11月出生,江苏省扬州人。
\end{authorinfo}
% 论文作者教育经历列表,按日期从近到远排列,不包括将要申请的学位。
\begin{education}
\item[2007年9月 --- 2010年6月] 南京大学计算机科学与技术系 \hfill 硕士
\item[2003年9月 --- 2007年6月] 南京大学计算机科学与技术系 \hfill 本科
\end{education}
% 论文作者在攻读学位期间所发表的文章的列表,按发表日期从近到远排列。
\begin{publications}
\item Dongxu Hou, Kanglian Zhao, Wenfeng Li ``An Extensible Emulation and Network Performance Analysis System for Space Information Network,'' in \textsl{IEEE GLOBECOM 2020 Conference Proceedings}, Dec. 2020.
\item Xiaobao Wei, Shiba Mao, Jinnan Chen, ``Protecting Source Location Privacy
  in Wireless Sensor Networks with Data Aggregation,'' in \textsl{Proc. 6th
    International Conference on Ubiquitous Intelligence and Computing (UIC)
    2009}, Oct. 2009.
\end{publications}
% 论文作者在攻读学位期间参与的科研课题的列表,按照日期从近到远排列。
\begin{projects}
\item XX十三五装备预先研究重点项目,“XXXX半物理仿真评估技术”,项目时间:2017年9月-2020年12月,军口预研纵向,已结题,主持。
\end{projects}
\end{resume}

%%%%%%%%%%%%%%%%%%%%%%%%%%%%%%%%%%%%%%%%%%%%%%%%%%%%%%%%%%%%%%%%%%%%%%%%%%%%%%%
% 生成《学位论文出版授权书》页面,应放在最后一页
\makelicense

%%%%%%%%%%%%%%%%%%%%%%%%%%%%%%%%%%%%%%%%%%%%%%%%%%%%%%%%%%%%%%%%%%%%%%%%%%%%%%%
\end{document}
